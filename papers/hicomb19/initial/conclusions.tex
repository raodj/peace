\section{CONCLUSIONS}

Clustering is a vital processing step for analysis of genomic data.
Several software tools have been proposed to enable fast clustering.
Nevertheless, continued advancement in clustering methods is necessary
to keep pace with the ongoing exponential growth of genomic data.
This study proposed two novel enhancements to an existing clustering
software system called \peace.  The first enhancement was the use of
Approximate Spanning Tree (AST) which enables \emph{much faster}
clustering than the current Minimum Spanning Tree (MST) approach.  In
addition, a novel Prime Numbers based Heuristic (PNH) is also
proposed.  The paper discussed these two enhancements in detail and
presented an empirical analysis of their effectiveness.  The empirical
data was collected from experiments conducted using two different
types of data sets.  Moreover, the paper also presented comparison
against \q{wcd-kaboom}, a \emph{fast}, state-of-the-art clustering
software.

The outcomes of this study show that the AST approach effectively
increase the performance of \peace\/ only for datasets with large
clusters (viral genomic sequences).  In the case of Influenza data
set, a dramatic 550\texttimes\/ performance improvement was observed.
In general, the AST approach did not significantly compromise the
quality of the clustering.  In most cases, the AST method generates
additional clusters, but did not impact purity.  These additional
clusters can be merged, if necessary.

The outcomes also indicate that the Prime Number based Heuristic (PNH)
is a promising approach for further increasing the speed, but only for
clustering long genomic sequences.  A detailed investigation of the
effects of hyperparameter choices for PNH is underway to increase its
broader applicability.  Furthermore, we are planning enhancements to
reduce memory footprint by using multi-tier heaps~\cite{higiro-17}.
Importantly, we plan to explore parallel clustering capabilities of
\peace\/ in conjunction with AST and the PNH.

Rapid clustering of viral and bacterial genomes forming large clusters
will provide insight into the genetic diversity. Viral genotypes such
as Influenza A subtypes can be identified by clustering unidentified
genomic sequences with known genotypes. This type of cluster-based
identification approaches has the potential to detect novel viral
strains by separating them into new clusters which does not consist
any known types.  The improved performance of \peace\/ with AST, with
its default parameter settings and without the need for additional
external tools (such as \q{mkesa} for \q{wcd}), makes it an ideal tool
for use by biologists for clustering large datasets.

\newpage
