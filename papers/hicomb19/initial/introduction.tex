\section{INTRODUCTION}

Clustering nucleotide sequence data has many applications ranging from
identification of gene expressions~\cite{hazelhurst-11}, reducing the
runtime of sequencing reads to the
genomes~\cite{rao-10,hazelhurst-08}, and inferring phylogenetic
relationships of organisms~\cite{cybis-18,solovyov-09,wei-12}.
Clustering approaches are being widely used to group or ``bin'' the
reads belonging to a taxonomic group together in metagenomics
analyses~\cite{sedlar-17}. Clustering approaches can be used to reduce
sequencing errors by pre-clustering reads in the initial stages of
data analysis~\cite{kozich-13}.  Clustering whole large sets of
chromosome, genomes and other larger nucleotide sequences using
alignment-free methods enable rapid identification of phylogenetic
relationships between organisms. An example of such an application
would be the clustering of large sets of viral and other microbial
genomes~\cite{delgado-15,solovyov-09,wei-12}. Clustering such
microbial data enables understanding the microbial phylogenetic
distribution in large datasets and potentially identifying novel
microbial taxa~\cite{wei-12}.

\subsection{Current state-of-the-art}

Classical sequence clustering approaches were based on sequence
alignments using dynamic
programming~\cite{morrison-15,sievers-14,thompson-03}. The major
drawback of alignment-based methods is runtime and memory consumption
-- \ie\/ they tend to be very slow and consume a lot of
memory~\cite{zielezinski-17}. Hence they are not preferred when
clustering large datasets. Moreover, even though alignment-based
approaches are considered to be highly accurate, the quality of the
results could be questionable due to several deficiencies as discussed
by Zielezinski \etal\/ (they specifically discuss five
cases)~\cite{zielezinski-17}.

%%They discusses five cases where alignment-based sequence analysis can
%%be deficient, including the inability to incorporate certain complex
%%sequence features such as duplication,
%%inversions~\cite{zielezinski-17} and secondary structural
%%features~\cite{morrison-15}.

Alignment-free approaches are more practical alternatives to dynamic
programming-based methods.  Instead of dynamic programming,
alignment-free approaches rely on partial comparisons and
pseudo-metrics for clustering.  For example, our preliminary
clustering software called \peace\/ used a well established
alignment-free, pseudo-metric called \q{d2} to estimate similarity or
``distance'' between pairs of reads.  \peace\/ uses the \q{d2} score
(more details in Section~\ref{sec:background}) to build a Minimum
Spanning Tree (MST) and then cuts the MST to form clusters.  \q{d2} is
also used by \q{wcd}~\cite{hazelhurst-08}.  Recently, Hazelhurst
\etal~\cite{hazelhurst-11} further enhanced \q{wcd} with a suffix
array based approach called \q{kaboom}, to significantly improve the
performance of \q{wcd}.  \q{wcd-kaboom} has shown to outperform (both
in runtime and cluster-quality) several mainstream clustering
software, including ESTate, xsact, PaCE, CAP3, and
TGICL~\cite{hazelhurst-11}.  Consequently, we use \q{wcd-kaboom} as
the reference for performance comparisons.

\subsection{Motivation for this research}

The primary advantage of alignment-free methods is that they run
\emph{much faster} and often have a small memory footprint but at the
cost of some degradation in clustering quality.  Consequently,
alignment-free methods are widely used for clustering large
datasets~\cite{zielezinski-17,vinga-14}.  However, the ongoing
exponential growth in data volumes, continue to pose challenges in
accomplishing fast and effective clustering.  Even with just ~1\%
pairwise comparisons~\cite{hazelhurst-11}, \q{wcd-kaboom} takes
\mytilde 149 minutes (on an Intel
Xeon\textsuperscript{\textregistered}\/ Gold 6126 CPU @ 2.6 GHz) to
cluster \mytilde65K Haemagglutinin (HA) reads.  Such a long runtime of
\mytilde 3.5 hours for a relatively small data set, with a
state-of-the-art tool, highlights the ongoing challenges, thereby
motivating the need for high-performance clustering solutions.

\subsection{Proposed solution: An overview}

This research proposes a high-performance clustering solution to
effectively cluster large datasets of both short and long genomic
data.  The objective is to reduce runtime without significantly
degrading clustering-quality.  We propose to improve performance using
two approaches: \ding{182} first we focus on improving performance by
changing the Minimum Spanning Tree (MST) clustering approach used in
\peace\/ with an Approximate Spanning Tree (AST), and \ding{183}
adding a Prime-Number based Heuristic (PNH) based on the previously
proposed prime number based scoring system for inverted repeats
detection~\cite{sreeskandarajan-14}.  We discuss the conceptual
underpinnings and algorithmic details of these two approaches in
Section~\ref{sec:ast} and Section~\ref{sec:pnh}.

The paper presents results from experiments conducted using a broad
spectrum of data sets in Section~\ref{sec:results}.  We also compare
performance of the proposed method with \q{wcd-kaboom} to highlight
the effectiveness of AST and PNH.  For example, with AST, the runtime
for clustering the HA dataset is just 16 seconds, instead of 8,940
seconds for {\ttfamily wcd-kaboom}, a 550\texttimes\/ performance
improvement.  This is a very substantial performance improvement with
clustering quality slightly better than {\ttfamily wcd-kaboom}.  This
study establishes the potential of the AST approach in providing a
high-performance clustering tool well suited for clustering datasets
with large clusters.  The PNH is a promising approach for further
increasing the speed of large sets of viral genomic data.

% It is hypothesized that the prime number-based filter will be very
% efficient in speeding up the clustering of viral genomic sequences in
% comparison to the Khaboom filter of WCD due to nature of viral genomic
% data: longer sequences, high similarity and few clusters.


%% For example, a popular Next Generation Sequencing technology Illumina
%% HiSeq2500 system is able to produce 100 nucleotides long reads, while
%% Helicos’s Heliscope is able to produce very short reads of average
%% length 30 nucleotides. The RSII platform of Pacific bioscience has the
%% ability to produce large reads of length more than 10,000
%% bases. Alignment-free clustering methods can aid the analysis of
%% sequencing reads data.

%% With the recent availability of large volumes of pathogenic viral
%% genomic data clustering tools based on alignment-free methods can aid
%% in the rapid analysis of pathogenic viral diversity and potentially
%% aid in the identification of newly emerging viral pathogens. Influenza
%% genome sequences are publically available via databases such as
%% Influenza Research Database~\cite{zhang-17}, GISAID~\cite{bogner-06}
%% and Influenza Virus Resource~\cite{bao-08}. Several other pathogenic
%% viral genomic datasets such as genomic sequences of Ebola, Dengue, and
%% Zika can be obtained from Virus Variation Resource~\cite{brister-14}
%% and Virus Pathogen Database~\cite{pickett-12}. Alignment-free
%% clustering approach enable effective clustering these large public
%% datasets aiding in the understand of the evolution of viral pathogens.


%% Based on testing performed of the modified version PEACE (PEACE 2) on
%% short nucleotide reads and large viral genomic the modified version
%% PEACE (PEACE 2) was found to be faster than WCDExpress and PEACE,
%% without much compromise in the accuracy of the clustering. Overall,
%% the AMAST modification was able to increase the speed of PEACE greatly
%% for sequencing reads clustering and viral genome clustering, while the
%% PNF was able effectively increase the speed effectively only for the
%% viral genomic data when combined with the AMAST. 
