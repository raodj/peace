\begin{abstract}
  \underline{Motivation}: Clustering genomic data, including those
  generated via high-throughput sequencing, is an important
  preliminary step for assembly and analysis. However, clustering a
  large number of sequences is time-consuming. \underline{Methods}: In
  this paper, we discuss algorithmic performance improvements to our
  existing clustering system called \peace\/ via the following two new
  approaches: \ding{182} using Approximate Spanning Tree (AST) that is
  computed \emph{much faster} than the currently used Minimum Spanning
  Tree (MST) approach, and \ding{183} a novel Prime Numbers based
  Heuristic (PNH) for generating features and comparing them to
  further reduce comparison overheads. \underline{Results}:
  Experiments conducted using a variety of data sets show that the
  proposed method significantly improves performance for datasets with
  large clusters with only minimal degradation in clustering
  quality. We also compare our methods against \q{wcd-kaboom}, a
  state-of-the-art clustering software.  Our experiments show that
  with AST and PNH underperform \q{wcd-kaboom} for datasets that have
  many small clusters.  However, they significantly outperform
  \q{wcd-kaboom} for datasets with large clusters by a conspicuous
  \mytilde 550\texttimes with comparable clustering quality.  The
  results indicate that the proposed methods hold considerable promise
  for accelerating clustering of genomic data with large clusters.
\end{abstract}

\begin{IEEEkeywords}
  Clustering, Minimum Spanning Tree, d2
\end{IEEEkeywords}

