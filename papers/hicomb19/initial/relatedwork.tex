\section{RELATED WORKS}

Many alignment-free sequence clustering tool have been proposed to
cluster large sets biological sequences. CD-Hit~\cite{li-06},
WCD~\cite{hazelhurst-08}, \peace~\cite{rao-10}, and
MeShClust~\cite{james-18} are some good examples for alignment-free
clustering tools. The popular clustering tool CD-Hit has the ability
to cluster large sets of nucleotide sequences, while WCD and \peace\/
are less fast but more accurate sequence clustering
alternatives. CD-Hit clusters sequences using a distance measure based
on the minimum number of shared short strings (words) between two
sequences.  WCD and \peace\/ use a similar word-based sequence
comparison approach but differ from CD-Hit in the distance measure
used to compare two sequences. Both WCD and \peace\/ utilize the d2
distance measure. \peace\/ differs from WCD by using a novel Minimum
Spanning Tree (MST) based clustering approach where the tree branches
are weighted by d2 distances.  It has been shown that the speed of WCD
can be improved by the use of a filter-based preprocessing
approach. The proposed suffix array-based filter for WCD named Kaboom
was shown to make WCD much faster than \peace~\cite{hazelhurst-11}.
Essentially, the Kaboom filter significantly reduces the number of d2
comparisons to be performed, to \textless\/ 1\% in many cases.  The
filter processing runtime is amortized by reducing the comparatively
slower d2 comparison.  However, the Kaboom filter requires generation
of suffix trees using a separate utility called \q{mkesa}.  MeShClust
is a recently proposed tool which claims to cluster nucleotide
sequences with high accuracy and considerable speed~\cite{james-18}.

The popular tools CD-Hit and WCD are primarily made to cluster
expression data such generated by sequencing methods as expressed
sequence tags and DNA sequence reads. They are shown to be well suited
for clustering next-generation sequencing data and other similar data
consisting of short length sequences. The utility of CD-Hit and WCD in
clustering whole chromosomes and genomes (which are larger in length
than typical sequencing reads) is questionable and has not been
extensively tested. Even though MeShClust claims to be efficient in
clustering short sequencing reads and relatively larger whole genomes,
it was not tested extensively for its whole chromosome/genome and long
read clustering ability.

Minimum Spanning Tree (MST) is a widely used method in clustering
nucleotide sequences and inferring phylogenetic relationships between
the sequences. Examples of such uses of MST include clustering
nucleotide sequences to aid in sequence domain
searching~\cite{guan-98}, performing intraspecific phylogenetic
analysis~\cite{rohl-99} and inferring phylogeographic
relationships~\cite{baric-03}.  A major issue associated with the MST
is the time complexity of the tree generation. The MST generation has
the time complexity of $O(n^{2})$ (where n is the number of reads to
be clustered)~\cite{caiming-13}.  Several approximate MST models
approaches have been proposed to reduce the time complexity using
heuristics approaches.  An example of an approximate MST approach is
minimizing the comparison overhead by building MSTs on K-Means
partition of the considered dataset~\cite{caiming-13}.  Another
example is the use of a novel centroid-based nearest neighbor rule for
the fast approximate MST generation (with the time complexity of $O(n
^{\frac{3}{2}} log\,n)$~\cite{jothi-18}. Employing such heuristic
approaches can aid in the development of efficient MST-based
nucleotide clustering and phylogenetic tools~\cite{rohl-99}.
