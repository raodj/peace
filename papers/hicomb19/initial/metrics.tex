\section{ASSESSMENT METRICS}\label{sec:metrics}

This study primarily focuses on reducing runtime for clustering.
However, the quality of clustering is important~\cite{hazelhurst-11}.
In this study we have used two widely used metrics for assessing the
quality of clustering, namely Normalized Mutual Information (NMI) and
Purity.  NMI is a popular measure of clustering quality based on
Information Theory~\cite{strehl-03}. Values of NMI range from 0 to
1. High NMI values indicate good clustering.  In this work, NMI is the
primary measure of clustering quality.

In addition to NMI, purity has also been used as a measure to further
validate clustering quality. Purity values range from 0 to 1.0.
Higher purity values indicate that a cluster does not have reads from
other clusters mixed into it.  The R packages aricode
(\url{https://cran.r-project.org/web/packages/aricode/index.html}) and
IntNMF~\cite{chalise-17} have been used to calculate NMI and purity,
respectively.

The NMI and purity values have been calculated using the clustering
output from \peace\/ as the reference.  Using the output from \peace\/
is motivated by two reasons.  First, \peace\/ has been tested by
multiple authors and has shown to produce good quality
clustering~\cite{rao-10,hazelhurst-11}.  Second, our objective is to
improve the performance of \peace\/ without impacting clustering
quality.  Consequently, we have used the output of \peace\/ as the
reference to assess clustering quality.

